\documentclass[journal,12pt,onecolumn]{IEEEtran}
\usepackage[utf8]{inputenc}
\usepackage{amsmath}
\usepackage{graphicx}
\usepackage[export]{adjustbox}
\title{Frequency Modulation}
\begin{document}
	\maketitle
	\section{Introduction}
	Generally the message signal isn't suitable for transmission as its frequency is quite low. Hence FM is used to modulate the message signal onto a higher frequency carrier wave which is suitable for transmission through the medium(air).
    The carrier wave is a sine wave with amplitude, phase and frequency. Different modulation techniques are available each of which changes one aspect of the carrier wave so as to encode the message signal onto it. Frequency modulation as its name suggests changes the frequency of the carrier wave based on the amplitude of the message signal.
\begin{equation*}
     $$s(t) = a(t)Sin(2\pi f_{c}t+ i(t)m(t)) $$
	\end{equation*}
	\newline
 \newline m(t) is the input signal
 \newline s(t) is output signal (modulated signal)
 \newline a(t) is the amplitude of the modulated signal
\newline i(t) is the modulation index

\textbf {Advantages of FM over Amplitude modulation (AM):}
\begin{itemize}
\item Less signal to noise ratio
\item Higher bandwidth and hence better quality audio signal
\item Less interference with transmissions in neighbouring channels.
\end{itemize}
	
\textbf {Disadvantages of FM:}
\begin{itemize}
\item Lesser range due to higher frequency.
\item More expensive equipment needed to transmit and receive.
\end{itemize}
\section{Block Diagram}

The input signal, from the microphone is passed through an amplifer. This is then frequency modulated onto the carrier wave generated by an LC circuit. Finally this modulated signal is sent for transmission through the antenna.
\begin{figure}
		\includegraphics[width=0.5\textwidth]{fm_block.jpg}
 		\end{figure}
 \section{Components Rquired}
 \begin{enumerate}
 	\item Resistors - 4.7K\Omega , 470 \Omega
 	\item Capacitors - 3.3pF, 1nF, 22nF
 	\item Variable Capacitor - 4pF - 40pF
 	\item Inductor - 220nH
 	\end{enumerate}
\begin{figure}[t]
		\includegraphics[width=0.7\linewidth]{transmitter.jpg}
 		\end{figure}
 		\section{Working}
 		Capacitor C1 is used to filter the input, and remove DC noise. the transistor in common-emitter configuration is used to amplify the signal. The LC circuit, generates the carrier wave and the amplified signal is modulated on the carrier wave. The carrier wave frequency changes based on the amplitude of the message signal. The modulated signal is passed through an antenna which produces electromagnetic waves (with frequency equal to carrier frequency - frequency of the oscillator) which are transmitted. 
 		\newline
 		\newline
 		\newline
 		\newline
 		\newline

\pragraph{\textbf {Team Members}}
	\begin{itemize}
		\item Jeel Bavsar - ES16BTECH11005
		\item Aravind Ganesh - EE16BTEECH11026
		\item Sri Laxmi Ganna - EE16BTECH11008
		\item Karthik - ES16BTECH11014
	\end{itemize}

	\end{document}
